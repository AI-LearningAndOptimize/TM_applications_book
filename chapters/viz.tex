

\chapter{Visualization}
\label{ch:viz}

Many of the following chapters are not just about algorithms using the
results of topic models: humans need to be in the loop somewhere.

Thus, topic models need to be shown to users.  We first talk about how
to show individual topics and then how those topics are presented.

\section{Displaying Topics}

As discussed earlier, topics are typically presented as a list of
ordered words.

However, there are relationships between words that are hidden in this
presentation.

Multi-word expressions can be discovered through
pre-processing~\citep{talley-11}, post-processing step~\citep{blei-09b},
or a joint model~\citep{johnson-10}.

Visualizations can incorporate information about probabilities to
highlight the most probable words (Word clouds).

But word clouds place words randomly, which can lead to spurious
associations.  Another alternative is to use word associations to
layout words~\citep{Smith:Chuang:Hu:Boyd-Graber:Findlater-2014}.

\section{Displaying Models}

But a topic model is more than about individual topics.

It is important to show the most relevant documents for reach
topics~\citep{chaney-12}

More sophisticated techniques can give the relationship between meta
data and topics~\citep{gardner-10,eistenstein-14}

It is also important to show the similarity between
topics~\citep{chuang-12}

Showing the relationships between multiple models can also help
distinguish stable from spurious topics~\citep{chuang-15} 

\section{Interaction}

But not all topic models are perfect.  Visualizations can help show
users where topic models have issues. 

One approach is to provide probabilistic constraints~\citep{hu-14:itm}

Another approach is to add matrix constraints~\citep{choo-13}

These interactions and visualizations allow users to discover and
refine insights, allowing them to explore and understand diverse
datasets.