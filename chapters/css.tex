
\chapter{Computational Social Science (jordan)}
\label{ch:css}

 while the previous chapters were mostly retrospective analyses, computational social science is mostly in the "here and now".

 traditional social science methods like polling do not scale well. And still require a lot of resources and time.

\section{Sentiment Analysis}

Modeling sentiment is an important problem for companies, politicians,
and anyone discussed online~\cite{pang-08}. 

% Find original citation 
Topic modeling can help by breaking large corpora into domains
(Surprising is bad for steering but good for books).

Sentiment is an example of meta data, which can be visualized to
better understand a corpus (see viz).

\section{Upstream and Downstream Models}

Better models, however, can jointly model meta data and sentiment.  

The advantage of joint models are both in the topics and for
prediction.

There are two broad categories of these joint models: upstream and
downstream models.

Upstream models can be supervised~\cite{mimno-08} or
unsupervised~\cite{lin-09}.

Upstream models are easier and more flexible~\cite{stewart-14}.

Downstream models provide better predictions~\cite{blei-07b} and
predictions can be better still with hinge loss~\cite{zhu-09}

These models form the foundation for the models and problems we
discuss in the rest of this section. 

\section{Understanding Stance and Polarization}

Often, discussions on an issue can break down into two sides.

Upstream models can discover these sides~\cite{paul-10}

So can downstream models~\cite{nguyen-13:shlda}. 

However, there are not always two sides to an issue.

A probabilistic solution to this model is the nested Dirichlet
process~\cite{blei-07}.

These hierarchies induce a non parametric hierarchy over an unbounded
number of topics 

This corresponds to agenda setting from political science~\cite{Nguyen:Boyd-Graber:Resnik:Miler-2015}

\section{Topic Models for Understanding Populations}

Traditional social science methods are labor intensive, take a long
time, or are impossible for sensitive subjects

For instance, surveys of influenza take too long to be useful compared
to the life cycle of influenza's progression 

Monitoring pollution in China or drug use in teens requires access to
populations that may be difficult.  Using social media presents an
alternative 

\section{Social Networks}

We have talked about meta data that are independent for each user.
Sometimes, however, we are interested in meta data that describe the
relationships between people 

This makes modeling more difficult, but we still see the same division
between upstream and downstream models 

The stochastic block model is the prototype for upstream models

Link LDA is the exemplar for downstream models 

Hybrid models can have the best of both worlds 

Supervised LDA bases regressions on the topic assignments rather than
the allocations

% journal version?
Doing something similar can also improve link prediction~\cite{chang-09a}

This can discover geographic variation in language

But what if we are interested in regions with multiple languages or
dialects? 
