\chapter{Information Retrieval (yuening)}
\label{ch:ir}

This section introduces the applications of topic models in information retrieval. We first discuss about the semantic relations between queries and documents, which can be extracted from topic models. Two main directions have been explored to apply topic models into information retrieval framework: one direction is to compute semantic similarity for semantic ranking relevance feature, and the other is to use topic models to smooth the language models. Some other approaches such as query and document reformulation are also briefly discussed.

\section{Semantic Relations}

\begin{itemize}
\item Brief introduction about information retrieval
\item Semantic relations between queries and documents: what and why
\item Existing methods and challenges
\end{itemize}

\section{Semantic Relevance}

\begin{itemize}
\item Latent Semantic Analysis and Probabilistic latent semantic analysis
\item Compute Semantic similarity for semantic ranking relevance
\end{itemize}

\section{Language Modeling}

\begin{itemize}
\item Language modeling framework for information retrieval
\item Applying topic models to improve the language models for information retrieval
\item Applying topic models to smooth the language models for information retrieval
\end{itemize}

\section{Other Approaches}

\begin{itemize}
\item Query reformulation
\item Document reformulation
\item probably more...
\end{itemize}



%e-discovery
%Transition to next section: what if you care about recall and understanding