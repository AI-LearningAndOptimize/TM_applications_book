
\chapter{Introduction}
\label{ch:intro}

Interacting with large text data sets is often posed as a needle in a
haystack problem: where is the document that says X~\citep{Salton-68} or the answer to
question Y~\citep{Hirschman-01}.

In contrast, topic models are about understanding large document
collections.  This is common in science policy~\citep{talley-11}, journalism, and the
humanities~\citep{moretti-13}.

In the same way that humans need organizational schema for large
document collections, so do computers.

Topic models provide a framework for understanding these document
collections for both humans or computers.

This survey explores the ways that humans and computers make sense of
document collections.

\section{What is a Topic Model}

Topic models get their name from the topics they discover from text.
We focus on an example from a corpus of scientific literature.

Examples of topics.

Conventions of labeling topics.

Allocations of documents to the topics.

\section{Foundations}

Topic models began with a linear algebra
approach~\citep{deerwester-90}.  While these approaches have seen a
resurgence in recent years~\citep{anandkumar-12:lda,arora-13}, we focus
on probabilistic approaches, which are intuitive, work well, and allow
for easy modification (as we see later in \dots).

Probabilistic models begin with a generative story.

The prototypical topic model is latent Dirichlet
allocation~\citep{blei-03}.  Dirichlet distributions encode sparsity
and the intuitions about how documents are formed and how works are
used.

Topics are distributions over words.

Document allocations are distributions over topics.

The generative story is a fiction~\citep{box-87}; it may not be needed
to understand the organization of document collections.

\section{Inference}

Distinction between prior and posterior.  You can skip this section if
you want.

There are many flavors of algorithms for posterior inference: message
passing~\citep{zeng-13}, variational inference~\citep{blei-03}, gradient
descent~\citep{hoffman-10}, and Gibbs sampling~\citep{griffiths-04}.

We focus on Gibbs sampling, which is simple, intuitive, and
fast~\citep{yao-09}.

We present the results of Gibbs sampling without derivation, which are
well described elsewhere.

Gibbs sampling begins with random initialization.

You then change the topic assignments for each word. Example based on
our scientific dataset.

From these data, you can then get the topics and the document
allocations.

\section{Implementations}

Many implementations are available.

Mallet is fast and widely used~\citep{mallet}.  This is where you
should probably start, in our biased opinion.

Variational inference is the other major option, but requires more
preprocessing~\citep{blei-03,vw}.

Other options require clusters~\citep{Narayanamurthy-11,zhai-12}.

Other frameworks allow you to specify arbitrary generative
models~\citep{stan-software:2014}.

\section{Rest of this Survey}

Brief discussion of the rest of the survey.

The next chapter returns to the distinction between high level
overviews and finding a needle in a haystack.  We show how a high
level overview can help users and algorithms find documents of
interest.