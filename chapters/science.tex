

\chapter{Understanding Scientific Publications}
\label{ch:sci}

Understaning scientific publications is important for funding
agencies, lawmakers, and the public.

Rigid classifications are difficult because fields
change~\citep{szostak-04}.

\section{Understanding Fields of Studies}

Topics can correspond to rough fields of studies.  This insight was
noted early in the development of topic models~\citep{griffiths-04}.

Topic models can show where official designations or labels conflict
with reality~\citep{talley-11}.

Where we have labels we trust, we can use them to constrain
topics~\citep{ramage-09}.

But if the labels are two coarse, we can use topic models to refine
them or organize the\citep{Nguyen:Boyd-Graber:Resnik:Chang-2014}

\section{How Fields Change over Time}
\label{sec:science_fields}

One way that science is unique from the fields discussed in the
previous chapters is that it is part of a continuous dialog.  Each
paper in its own way stands on the shoulders of giants. Topic models
for science thus need to be aware of the connections between documents
over time.

One of the first techniques to do this viewed topics as subtly
changing each year~\citep{blei-06b}.  Example of physics going from
ether to accelerators.

The flipping of a calendar page does not rule science, however;
changes can happen at any time~\citep{wang-06,wang-08}.

\section{Innovation}

Fields change because innovation happens.  Identifying who is
responsible for the changes can find who is ``winning'' the race to
introduce, explain, and popularize new ideas.

From an institutional perspective, we can see which universities have
research portfolios that look like the future~\citep{ramage-10}.

This is important not just for science historians but also for policy
makers~\citep{largent-12}.

Institutions are where science happens, but the true drivers of
innovation are individual researchers who present their papers as the
place where research happens~\citep{gerrish-10}.
